\documentclass[11pt,twoside]{article}
\usepackage[toc,page,header]{appendix}
\usepackage{pdfpages}
\usepackage{csquotes}
\usepackage{changepage}
\usepackage{fontspec}
\defaultfontfeatures{Scale=MatchLowercase}
\setmainfont[Mapping=tex-text]{Times New Roman}
\setsansfont[Mapping=tex-text]{Arial}
\setmonofont{Courier}

\usepackage{float}
\usepackage{turnstile}
\usepackage{bussproofs}

\usepackage{geometry}
\geometry{letterpaper}

\newtheorem{theorem}{Theorem}
%\newtheorem{cor}{Corollary}
%\newtheorem{lem}{Lemma}
%\theoremstyle{remark}
\newtheorem{remark}{Remark}

\newtheorem{objection}{Objection}
\newenvironment*{response}[1][]{\noindent
\textbf{Response to Objection #1.}
\begin{adjustwidth}{1em}{1em}
}
{\end{adjustwidth}
\vspace{1ex}
}


%\usepackage[parfill]{parskip}    % Activate to begin paragraphs with an empty line rather than an indent

\usepackage{graphicx}
\usepackage[leftcaption]{sidecap}
\sidecaptionvpos{figure}{c}

%\usepackage{amssymb}

\usepackage{epstopdf}
\DeclareGraphicsRule{.tif}{png}{.png}{`convert #1 `dirname #1`/`basename #1 .tif`.png}

\usepackage[
bibstyle=numeric,
citestyle=authortitle-ibid,
natbib=true,
hyperref,bibencoding=utf8,backref=true,backend=biber]{biblatex}

\usepackage{hyperref}
\hypersetup{
    bookmarks=true,         % show bookmarks bar?
    unicode=true,          % non-Latin characters in Acrobat’s bookmarks
    pdftoolbar=true,        % show Acrobat’s toolbar?
    pdfmenubar=true,        % show Acrobat’s menu?
    pdffitwindow=false,     % window fit to page when opened
    pdfstartview={FitH},    % fits the width of the page to the window
    pdftitle={Deflating Validity},    % title
    pdfauthor={G. A. Reynolds},     % author
    pdfsubject={Validity},   % subject of the document
    pdfcreator={G. A. Reynolds},   % creator of the document
    pdfproducer={G. A. Reynolds}, % producer of the document
    pdfkeywords={Validity} {Survey Research}, % list of keywords
    pdfnewwindow=true,      % links in new window
    colorlinks=true,       % false: boxed links; true: colored links
    linkcolor=blue,          % color of internal links
    citecolor=blue,        % color of links to bibliography
    filecolor=magenta,      % color of file links
    urlcolor=cyan           % color of external links
}
\usepackage{draftwatermark}


\usepackage{fancyhdr}

%% \setlength{\headheight}{15.2pt}
\pagestyle{fancy}
\fancyhead{} % clear header flds
\fancyhead[RO,LE]{Validity in \sr{}}
\fancyfoot{} % clear footer flds
\fancyfoot[RO,LE]{\thepage}
%% \chead[]{}
%% \rhead{Validity in \sr{}}[\thepage]

\title{Deflating Validity \\
\vspace{12pt}\large{or: The use and abuse of \enquote{validity} in \sr{}}}
\author{G. A. Reynolds}
\date{\today}
\bibliography{%
../bib/validity.bib,%
../bib/logic.bib,%
../bib/mind.bib,%
../bib/philosophy.bib,%
../bib/psychometrics.bib,%
../bib/measurement.bib,%
../bib/psychology.bib,%
../bib/variables.bib,%
../bib/val.bib,%
../bib/psychomet.bib}

%% Macros

\newcommand{\SM}{Standard Model}
\newcommand{\XSM}{Extended Standard Model}

\newcommand{\SMeth}{Survey Methodology}

\newcommand{\SR}{Survey Research}
\newcommand{\sr}{survey research}
\newcommand{\SRIV}{Survey Interview}
\newcommand{\sriv}{survey interview}
\newcommand{\SIV}{Survey Interviewing}
\newcommand{\FI}{Field Interviewer}
\newcommand{\Iver}{Interviewer}
\newcommand{\R}{Respondent}
\newcommand{\LPR}{Legal Permanent Resident}
\newcommand{\ART}{Assimilated Response Technique}
\newcommand{\GAM}{Grouped Answer Method}
\newcommand{\IOM}{Instrument of Measurement}

\includeonly{%
%% pilots,cards
}
%%%%%%%%%%%%%%%%%%%%%%%%%%%%%%%%%%%%%%%%%%%%%%%%%%%%%%%%%%%%%%%%
\begin{document}
\maketitle
\nocite{*}

\begin{abstract}
In philosophy and logic, validity and truth are closely related.
Truth is a property of sentences (propositions); validity is a
property of inferences.  In recent decades, ``deflationary'' (or
``minimalist'') accounts of truth have become increasingly popular
among philosophers.  Broadly speaking, these accounts deny that truth
is a substantial property, and instead treat the term ``truth'' as a
kind of expressive device; it adds nothing significant to the
expressions in which is appears, but it makes the language
significantly more powerful.  It allows us to say things we
otherwise could not say, or could only say in cumbersome ways.  For
example, with a locution like ``... is true'', we can endorse claims
by naming them (e.g. ``Fermat's last theorem is true''); without such
a locution, we would have to explicitly repeat the theorem as a claim
(e.g. ``There is no integer z greater than 2 such that ...'').  And
some things we can say with ``... is true'' would be practically
impossible to express without it, such as ``everything the policeman
said is true'' (since it would not be possible to repeat everything he
said) or ``the theorems of group theory are true'' (since there are (I
assume) infinitely many such theorems).

A third aspect: sentences contain referring components.  To the truth
of a sentence corresponds the ``referentiality'' of its components.
``Snow is white'' is true; it is true because ``Snow'' refers to the
famous cold stuff, and ``white'' refers to the famous color.  We need
(but generally speaking do not have) a technical term to refer to the
property of such refering relations that corresponds to the property
of truth of sentences.  It is a category mistake to say ```Snow' is
true'', but we would like to say ```Snow' is $x$'' in order to bring
attention to this truth-like referential condition.  In \SR{} (and
social science in general), the term ``validity'' is often recruited
to serve this need in measurement vocabulary.  The inadvisability of
this becomes obvious when you move from measurement to description:
``2.3 meters is a valid measurement of the length of x'' is a common
way to talk, but ```Snow' is valid'' sounds decidedly off-key.

This paper has two goals.  The theoretical goal is to do with validity
what deflationists have done with truth.  The more practical goal is to
examine the use and role of the concept (term) validity in \SR{}.

The first part of the paper thus explores the plausibility of a
deflationary or minimalist concept of validity.  Not just logical
(inferential) validity, but validity as used by the social sciences,
as a property of referential relations.

The second part of the paper examines the notion of validity as used
in \SR{}.  Suffice it to say that vocabulary of validity in the social
sciences, especially psychology and education research, is very,
\textit{very} confused.  Generally speaking, the term is used to
refer, not to inferences and their properties, but to referential
relations.  Classic definitions of validity in the social sciences
usually say something like ``measures what it purports to measure'',
which is to say, measurement expressions (e.g. ``2.3 meters'')
\textit{refer} to entities (properties, relations) in the world.  But
it is also used to refer to inferences and a variety of other
concepts.



The connection between the first and second parts is that the social
sciences usually treat validity as a substantial property.  Theories
of validity often take on a metaphysical hue; they attempt to say what
validity \textit{is}, as if it were some kind of entity or substance
-- validity stuff -- that referring terms ``have'', possibly in
greater or lesser degrees.  On the deflationary view, this is a
mistake that inevitably leads to unresolvable problems.

\end{abstract}

\tableofcontents
\listoffigures

\section{Deflationism}

\begin{remark}
  Deflationism seems to depend essentially on some form of
  expressivism.  Or maybe they amount to the same thing?
\end{remark}

\subsection{Validity, Reliability, Error}
\label{sub:Validity}

\begin{remark}
What is the point of worrying about validity?  Is it something in the
world that we are trying to discover?  Then we're trying to find ``the
right description of the world'' (Putnam).  Or is it a concept, so
that validity talk is about conceptual analysis and definition?

Or: we try to find the right description, and validity talk is part of
how we decide that we have found it.

\end{remark}

\begin{remark}
Why do psychometricians and the like worry so about validity?

Hypothesis: when they say ``validity'', what they're really interested
in is scientific legitimacy.  Effectively, to say that a test (etc.)
is valid is to say that it is in fact scientific.  Thats the practical
import of the concept of validity for them.

Unpack this.  Expose the assumptions and implications.
\end{remark}

\begin{remark}
  The problem with validity (quantifiability) is circularity.  If the
  task is to show that some property is quantitative, we have to do
  this without relying on quantitative vocabulary.  So for example, if
  we want to show that temperature is quantitative, we cannot use the
  concept of a unit of temperature to do so, because that presupposes
  just the outcome we are supposed to demonstrate.  This is similar to
  the problem we face in seeking to account for representational
  vocabulary in non-representational terms.

  quantifiability v. validity?  distinct problems, but the latter
  depends on the former?
\end{remark}

key concepts:

\begin{itemize}
\item validity treated as a special kind of property - of what?
\item constructs
\item (latent) variables
\item indicators
\end{itemize}

``validity'' as code for:

\begin{itemize}
\item legitimacy
\item vindication
\item credibility
\item proof (good premises + valid inference)
\end{itemize}

\begin{remark}
  On the idea that validity something (a property, etc.) that we look
  for in scientific theories in order to distinguish good ones from
  bad: see Putnam on fact/value distinction.  We use value judgments -
  simplicity, parsimony, etc. - in every aspect of science (thought),
  esp. in weeding out bad theories.  For there is no external or
  objective criterion of acceptability for theories to which we can
  appeal, nor is there any such citerion that does not involve value
  judgments.
\end{remark}

\begin{remark}
  So along with the fact/value distinction, and the analytic/synthetic
  distinction, the internal/external distinction also collapses?  Or
  do we just exclude the notion of external?  No; we need to retain
  the idea of an external world that is independent of us and to which
  some of our judgments are answerable.  We don't get to just make
  stuff up and call it true (correct) for at least some of our claims.
  There is no external absolute authority that can decide for us which
  theories are true, or rather which we should endorse, but that does
  not mean there is no external world that is authoritative for some
  of our sayings.  But isn't that trying to have it both ways?  How
  can our theories answer to the world if we cannot appeal to the
  world or some other external authority to sort them out?  See
  Brandom.

Related issue: what counts as evidence?  How do we decide?  What are
we doing when we decide that something counts as strong (weak)
evidence in support of a theory?  What are the criteria of adequacy
for an account of evidence?
\end{remark}

\subsection{RCT and Self-validation}

See Cartwright on RCT as self-validating.  This seems to mean that
RCTs are valid by construction.

This nicely parallels industrial QA notions of guaranteeing quality by
designing a production process that prevents defects.

What's the logic here?  Is self-validation really possible?  How can a
process validate itself - isn't the very idea inherently circular?  Or
rather, don't we land in a regress?  After all, if the idea is to
specify a process that yields validity, how do we know that that
process is itself valid?

\subsection{Deflation}

How can we get out of this mess?  One way is to deflate the notion of
validity, just deny that it is a substantive property.  When we claim
that a result is valid etc. what we are really saying is that we
endorse it, approve of it, etc.  It's an expressive device.  Compare
the semantic deflationist's idea that calling something true amounts
to endorsing or approving of it.

So if we discard the notion of validity (since it does no real work),
don't we find ourselves lacking something essential?  Well, we just
need a vocabulary that allows us to say explicitly the sorts of things
we find it useful to be able to express with respect to a study or qx
technique.  For example: credibility, utility, legitimacy,
vindication, justification, etc.

\begin{remark}
  The notion of validity seems to be connected to the problem of
  deciding which theories we should endorse.  What are the criteria of
  adequacy for any notion (or theory) of validity?  Or: what are the
  requirements that should be met by any purported explanation of
  validity?  Both particular cases and the general idea.  Tarski gives
  us something like this for logical validity; what about ``validity''
  as the term is used by psychometricians, test theorists, etc.?

Contrast: claims of validity for a case, v. explanation of what
validity is.


\end{remark}

The objection will no doubt be that we need some kind of standard,
which is just to say that we want to measure this something (validity,
credibility, whatever).  Implicit in all this is the notion that there
is some ``objective'' fact of the matter to which our
study/technique/etc. is ansswerable. A study is valid iff - what?  If
it meets some definite ``objective'' criteria.  Methodological
criteria, conditions of validity, etc.  In the psychometrics and
testing tradition this appeal to external authority is expressed as
something along the lines of ``measures what it purports to measure''.
Which is only meaningful insofar as a) there is actually something
there to measure, and b) it is in fact susceptibel to measurement.

And usually this is expressed in statistical terms.  But that dog
won't hunt either - you cannot get to validity via statistics.  All
you can do is measure central tendencies and variance - not enough to
establish validity, which is a substantive notion. (analysis
elsewhere).

To say that sth is valid is just to say that it is admirable
(Peirce?), or perhaps that it is virtuous, that it has the virtues we
prize.

%%%%%%%%%%%%%%%%
\section{Previous Work}
\label{sub:PreviousWork}

\begin{remark}
Fact/value; inference; 
\end{remark}

\subsection{Landmark articles}

Cronback \& Meehl, Messick, Lissitz, Michell, etc.

Special issues:

Social Indicators Research Volume 45, Issue 1-3, November 1998:  Special Issue on Validity Theory and the Methods Used in Validation: Perspectives from the Social and Behavioral Sciences

\noindent
\cite{schwarz_is_2009}

\subsection{Validity}

\subsection{Measurement}

\subsection{Assessment and Evaluation}

\noindent
\cite{brinkmann_factval_2005} \\
\cite{cicourel_interviews_1982} \\
\cite{hood_validity_2009} \\
\cite{mcdonald_measuring_2011} \\
\cite{messick_validity_1995} \\
\cite{putnam_collapse_2002}

%%%%%%%%%%%%%%%%
\section{Kinds of Validity}
\label{sub:ValidityKinds}

Validity of what?  Question, answer, both, or something else.

\begin{remark}
  The critical idea is that representational validity
  (\enquote{measures what it purports to measure}) is itself validate
  by causality.  What matters is that the measurand (attribute to be
  measured) should \textit{cause} change in the \IOM{}.

  Example: thermometer.  The readings of a (good) thermometer are
  \enquote{valid} measurements of temperature just because temperature
  (or heat or?) \textit{causes} them.  NOT because the recording
  measurements represent.  In other words, representation comes after
  causality.

  This seems to imply that validity is a property of the \IOM{}.
  Which in turn argues that we should come up with a different term.
  In any case, it is essential to distinguish between the causal
  relationship between measurand and \IOM{} on the one hand, and the
  behavior of the instrument and the measurement scale on the other,
  and on the third hand, the reading of the measurement and its
  recording as text in the proverbial lab notebook.
\end{remark}

Classic definitions: validity means it (test, etc.) measures what it
purports to measure.  This is not a very clear definition; it fails to
make the essential distinction between referentiality and accuracy,
etc.  Do they mean to say that e.g. an inaccurate thermometer is
valid, so long as it does measure temp?  What this means is so long as
its (inaccurate) readings are caused by temperature?  (But if they are
inaccurate there must be other causes at work.)

A better definition: a valid measurement is one that is caused by (and
only by) the measured property, regardless of accuracy.

But the standard definition also may be read as implying that
\enquote{measuring what it purports to measure} means that it measure
the true value.  But since all meassurements contain error, what can
this mean?  Seems \enquote{measure true value} must mean \enquote{is
  cause by the measured property}.

\begin{remark}
  This characterization of measurement in terms of representation is
  not to be confused with the Representational Theory of Measurement,
  which is but one of several competeing theoretical models of
  measurement.  A criterion of adequacy for any such theory is that it
  must be able to account for the representational aspect of
  measurement.
\end{remark}

The term is used in so many ways that it effectively means not much
more than \enquote{good, valuable}.  In the literature one can find
valid: questions, answers, data, etc. etc.  These are different
things; what property common to them all does \enquote{validity} name?
Is there such a common property?  If \enquote{good} is the best we can
do, then the concept of validity is largely trivial if not vacuous.

Need to distinguish between validity and accuracy.  An inaccurate
thermometer may not measure temperature accurately, but it does
measure (i.e. respond to) temperature.  So bad results are not
necessarily invalid, only inaccurate.

The classic concept is that validity is about what gets measured, or
whether the measurement indeed measures what it purports to measure.
Underlying this notion seems to be the idea that the measurement
outcome (the number on a scale) stands in a particular relation to the
measurand.  In other words, this is a representationalist notion:
validity is about goodness of representation, where representation is
viewed as a relation between signifier and signifiee.

Notice that such a concept of validity can only apply to attributes
that are in fact quantitatively structured (measurable).

Thus ``question validity'' would describe the relation between the
question and what it purports to measure.  But since questions do not
measure anything, the real idea seems to be that question validity is
about the (representing) relation between the question and what it
purports to represent.

Pragmatic alternative: good measurement useful, in that it allows us
to make good predictions.

Example: simple math test of one question: what is the sum of 1 and 1?
Possible correct answers: 2; square root of 4; difference between the
2nd and 3rd prime numbers; etc.  Not to mention: II (Roman numberals);
deux (French), dos (Spanish), etc.

Moral of the story:  q-a pairs do not measure.

Re: use of tests as a matter of validity (i.e. many uses, each of
which must be validated, so many validities, etc.).  Compare use of
temperature.  The financial authorities could set monetory policy
based on average temperature, if they wanted to.  That would be an
invalid (bad, incorrect, unjustifiable) use of temperature
measurement, but the use of temperature measurements has nothing
whatsoever to do with the \enquote{validity} temperature measurement.

The fact that different uses of social measurement do have something
to do with validity should be taken as a sign that such
\enquote{measurements} are not measurements at all, at least not in
the way temperature measurements are.

Why?  Because temperature \textit{causes} change in the instrument of
measurement.  What makes it valid is reference to temperature and only
temperature, which happens because temperature causes the reading.
Compare use of a barometer as a proxy for thermometer.  Why is this
invalid?  Because changes in the reading may be caused by factors
other than temperature, and changes in temperature do not
\textit{necessarily} cause changes in the reading.  So barometric
measurements do not refer to temperature -- \enquote{refer} meaning
something like \enquote{be directly caused by}.  Directly, because
temperature does (or at least can) cause a change in barometric
pressure, thus causing a change in the reading of a barometer, but the
causal link is indirect.  It causes a change in pressure, which causes
a change in temperature.

Compare this with any social measurement, e.g. how old are you, or
what is 2+2 or the like.  There is no direct causality involved.

That's in addition to the problem of whether putative latent variables
are in fact measurables (have quantitative structure, etc.)  Even if
we could find a convincing latent psychological variable and somehow
demonstrate that it is quantitative, we would still have no warrant
for treating questions and answers involving it as measurements.

The task of \sr{} is to investigate normative commitments and
entitlements, rather than causes and effects.  Deontic rather than
natural necessity.  The means of doing so is controlled, explicit,
deontic scorekeeping.

References:

\noindent
\cite{messick_validity_1995}\\
\cite{hood_validity_2009}

\subsection{Success Conditions}

\begin{remark}
The ideal is that we should say that some instrument "works", that is, that it has worked in the past and will work in the future.

"Validity" is used by the orthodoxy to convey this ideal notion - if a
survey instrument or procedure is "valid" then we can use it in future
research.

Validity is variously described as a property of instruments, or of
inferences involving "interpretation or use" of a test, etc. etc.

A better approach is to talk of success conditions rather than
validity.  In the case of the ART, we would like to claim that it
works, but what does this mean?  It works IF it is properly used, and
if various other things are done - e.g. if quex items preceding the
ART item are "right", if the project as a whole is "correctly"
presented to respondents, etc.

In short, what we should claim is not "ART works" tout court, but that
we have identified the conditions under which it is likely to work -
the success conditions.  Compare the notion of truth conditions as
giving the meaning of a proposition.  Instead of "Is P true?" we ask
"What are the truth conditions of P?" - what conditions make it true.
Similarly, instead of "Is the technique valid?" we ask "What are the
success conditions of the technique?" - what conditions make it valid.
\end{remark}

\subsection{Representational Validity}

\sr{} seems to derive its notions of validity from psychometrics, test theory, etc.

``The problem of validity is that of whether a test really measures
what it purports to measure.''  (Kelly quoted in Saint-Mont).

This seems to be pretty clearly a representationalist idea.

Note that it is also rather vague, since it does not address accuracy.
If all that is require is that a test measure (i.e. refer to) what it
purports to measure, then it is an all-or-nothing affair, just like
logical validity of inference.  There would be no question of degrees
of validity, and validity itself would not be measurable on a scale.

\subsection{Epistemic Validity}

I.e. truth-value of propositions.

\subsection{Logical Validity}

The logical concept of validity is relatively straightforward: it is a
property of inferences only.

\begin{remark}
  NB: distinction between rules of inference (schemata,
  generalizations), and particular inferences.  To keep this clear, we
  might use \enquote{formal inference} for the former and
  \enquote{vernacular inference} for the latter?
\end{remark}

\noindent
\cite{prawitz_inference_2009} \\
\cite{prawitz_epistemic_2011}

%%%%%%%%%%%%%%%%
\subsection{Material Validity}

A concept of \textit{material inference} was introduced by Wilfrid Sellars in x.

\subsection{Statistical Validity}

Statistical concepts of validity are pervasive in the \sr{} literature.\ldots

\subsection{Discursive Validity}

\enquote{Validation} of a question = exploration of proprieties of
discursive practices governing use of the question.

This is a substantially different approach from the orthodox approach.
Usually validation is taken to involve question semantics.  The
meaning of the question is often antecedently assumed, often based on
some variety of Gricean intentionality; the literature often discusses
this in terms of the respondents understanding of the
\enquote{intended meaning of the question.}

Pragmatic \enquote{validation} seeks instead to discover the
discursive \textit{norms} governing the performance of the question.
Since the question is construed in practical terms, as a discursive
turn, its meaning cannot be antecedently presumed.  The task is not to
determine whether the respondent understands the intended sense of the
question, but to discover the norms implicit in the way respondents
treat the question.  Questions establish conditions of correct
response; but those conditions may differ for every person, since they
depend on collateral commitments and entitlements.  The goal is to
explore this; only after we have gotten a grip on how respondents
treat a question can we move on to the question of how this
corresponds to the researcher's understanding (i.e. the intended
sense).

In other words, \enquote{intended meaning} does not establish a
standard against which respondent performance is to be judged.  That
would beg the question, which is just what the question means
(i.e. how is it normatively used).

\subsection{Example:  Age}

Suppose we want to validate the question ``How old are you?''

Most of us can respond appropriately to this question by citing a
number representing age in years.  But the correctness of this
response is a matter of discursive practice rather than
representational accuracy.  It counts as knowledge not because the
number cited represents age, but because we have been trained to
respond to the question appropriately.  We do not examine the number,
see that it represents our age, then therefore cite it; nor do we
examine our age (a latent variable?) and see that it is represented by
a number which we then cite.  We have been trained from an early age
to cite a number in response to questions of this sort, and to
increment that number by one on every birthday.  This is a matter or
normative discursive practice.  If we do not respond to the question
by stating our names or something else unrelated to age, that is
because doing so would violate the discursive norm, not because hidden
cognitive processes prevent us from doing so for purely semantic
reasons.

That is not to say that our adherence to such practical norms is a
matter of mere response to a stimulus.  To borrow a bit of imagery
favored by Brandom, a parrot can be trained to reliably respond to our
age question by producing a number noise, say \enquote{three}.  But it
will never age; in the absence of retraining, every year it will
announce the same age.  That's because its response is non-conceptual.
It does not \textit{understand} the noises it responds to nor those it
produces.  It presumably has no grasp of the inferential articulation
of the concepts of age, three, you, etc.  Furthermore it has no notion
of correctness; it cannot decide that ``three'' is incorrect and
therefore decline to respond.  And although one might concede that its
reliable responsive disposition is a kind of discursive practice
(since it involves turn-taking or sequencing), it does not qualify as
\textit{linguistic} discursive practice.

What's the significance of this example?  First, it elucidates the
distinction between non-conceptual and conceptual (that is, rational)
responsiveness in terms of grasp of inferential structure.

\begin{remark}
  But what we need is clarification of how our conceptual life is
  grounded in normative practice; do parrot examples do this?
\end{remark}

%%%%%%%%%%%%%%%%%%%%
\section{Causality and Normativity}

Norms make the correct answer \enquote{necessary} (i.e. obligatory),
but not compulsory.

So questions do not \textit{cause} responses.  They establish the
criteria by which the answer may be judged as correct or not, but they
do not causally \textit{compel} the correct response.

Furthermore, neither questioning nor answering is purely
individualistic.  They involve discursive practices that are
essentially social.  So the \R{}'s answer to a question cannot be
construed as something cut off from social relations, nor as
referencing a purely private state, believe, disposition, etc.  Since
meaning is socially constituted, statements always involve more than
the personal; they essentially involve, for example, positioning,
role, status, etc.



%%%%%%%%%%%%%%%%%%%%%%%%
\section{Measurement and Error}

%%%%%%%%%%%%%%%%
\subsection{Measurement}

\sr{} (at least as far as interviewing methodology is concerned) seems
to depend almost entirely on notions of measurement drawn from
psychology in general and \textit{psychometrics} in particular.  But
quantitative psychology, and especially psychometrics, has its
critics; one authority on measurement has gone so far as to claim that
psychometrics is a form of \enquote{pathological} science, arguing
that \enquote{[Q]uantitative psychology manifests methodological
  thought disorder}\footcite{michell_quantitative_1997}, and that
\enquote{[P]sychometricians are not only uncritical of an issue basic
  to their discipline, but that, in addition, they have constructed a
  conception of quantification that disguises
  this.}\footcite[639]{michell_normal_2000} The gist of his argument:

\blockquote[][]{Consider any attribute that psychometricians currently
  believe they are able to measure (such as any of the veraious
  intellectual abilities, personality traits or social attitutes that
  the textbooks mention), and ask the question, \textit{Is that
    attribute quantitative?}  The hypothesis that such an attribute is
  quantitative underwrites the claim to be able to measure it.
  However, \textit{there has never been any serious attempt within
    psychometrics to test such
    hypotheses.\footcite[648; \textit{emph. added}]{michell_normal_2000}}}


\begin{remark}
  TODO: concise description of our approach to measurement in the
  context of this research.  Summary: the answers provided by \R{}s in
  response to \Iver{} questions are not to be taken as proper
  \textit{measurements} of anything.  Nor are the data that the
  \Iver{} records and the analyst eventual manipulates to be
  considered quantified representations of measurable (quantifiable)
  properties.  Instead, answers are discursive turns or moves whose
  \textit{contents} are implicit propositions endorsed by the \R{},
  and the recorded data are representations of those endorsements.
  Measurement only enters the picture when the statistician describes
  the collection of recorded endorsed propositions.  In other words,
  the statistical description of the collected data serves to
  characterize what people assert (the propositions they endorse;
  their speech practices) rather than \enquote{levels} of hidden
  properties.
\end{remark}

\begin{objection}
  How does this apply specifically to our question about status?  Is
  it not correct to interpret a \R{}'s true correct answer to our
  question as a \enquote{measurement} of status?  On the orthodox
  view, the answer corresponds to the \R{}'s actual status; isn't this
  properly considered measurement?
\end{objection}
\begin{response}[1]
  A \R{} who truthfully claims e.g. \enquote{I am a \LPR{}} thereby
  signals a \textit{commitment} to the propositional content of the
  statement (namely, that R is a \LPR{}).  Any such commitment may be
  challenged, in which case the \R{}'s \textit{entitlement} to the
  claim comes to the fore.  The \R{} might \textit{justify} the claim
  by citing any number of \textit{reasons}, including personal
  knowledge, etc., or may exhibit non-linguistic evidence (such as an
  official LPR card).

  The critical point is that the \R{}'s answer is not \textit{caused}
  by anything; in particular it is not caused by some occult
  \enquote{variable}.  What matters to the meaningfulness of the
  discursive exchange is not a matter of natural causes, but of
  propositions, commitments, entitlements, authority, and the other
  aspects of discursive practice involved in \enquote{the game of
    giving and asking for reasons}.
\end{response}


\begin{objection}
  If that is the case, if individual answers are not in fact
  measurements of anything, then what becomes of the statistical
  analysis of the collected data?  Is it not measurement?
\end{objection}

\begin{response}[2]
  The collected data is just that, a set of data, so like any set it
  can be described statistically.  The critical question is what is
  the \textit{meaning} of that statistical description.  The orthodox
  approach claims that it measures something -- a latent variable or
  the like.  Implicitly at least, it claims that there is a
  determinate relation between the statistical description and the
  levels (or whatever) in the individual \R{}s.  The claim here is
  that this claim is unwarranted.

  The claim here is that the business of the \SIV{} is just deontic
  scorekeeping.
\end{response}

\noindent References:

\noindent
\cite{michell_normal_2000}\\
\cite{sherry_thermoscopes_2011}

See British Journal of Psychology, Aug 1997 vol 88 issue 3:
\cite{michell_quantitative_1997} and six commentaries.

%%%%%%%%%%%%%%%%
\subsection{Variables}

References:

\noindent
\cite{toomela_variables_2008}\\
\cite{schwarz_is_2009}\\
\cite{stam_fault_2010}

%%%%%%%%%%%%%%%%
\subsection{Error}
\enquote{Problems} in a \SRIV{} should not automatically be treated as
indications of \textit{error}, whether \enquote{\R{} error},
\enquote{\Iver{} error} or some other class of error.  Discrepancies
between \R{} understanding and researcher understanding should often
be treated as evidence of normal variance in how different members
treat the question.  \enquote{Problems} in question-answer
interactions may be traceable to mistakes, misunderstandings, etc.

\subsection{The Metrological Corkscrew}

The fundamental problem confronting any purported measurement is the
metric equivalent of the hermeneutical circle.

For a measure of an attribute to be \enquote{valid}, the attribute
must be measureable in the first place.  The problem is that the only
way to determine whether or not an attribute is quantifiably
measurable is to measure it.  Measurability cannot be antecedently
established.

Sherry's account of the history of the development of temperature
measurement\footcite{sherry_thermoscopes_2011} clearly illustrates the
problem.  This problem also lies at the heart of Michell's attack on
psychometrics\footcite{michell_normal_2000}.


\subsection{The \enquote{Referential Gap}}

By \textit{referential gap} I mean the conceptual chasm between the
meaning of statistical description and the meanings of the individual
data so described.

Consider a canonical example of measurement: using a thermometer to
measure temperature of a liquid.  An individual (specific occasion of)
measurement yields a number on a scale that corresponds to the
\enquote{true} temperature of the liquid.

Such an account is not particularly controversial\footnote{For most of
  us, anyway; the theory of measurement is quite subtle, and there are
  several different approaches, some of which may not agree with the
  emphasis I put on representation here.  Nonetheless, here we keep it
  simple for the sake of clarity.  The point of this toy example is to
  clarify the issues involved in order to show how measurement
  concepts in \sr{} are problematic and to make an alternative
  pragmatist concept intelligible.}, but there is much more than meets
the eye here.  First, this account assumes but does not demonstrate
that \enquote{true temperature} is meaningful; second, it takes
\enquote{valid} (or, good, or correct, etc.)  measurement as
essentially involving accuracy of representation or correspondence.

Let's assume for the sake of argument that things really do have a
true temperature, and that our thermometer accurately represents that
temperature.  In this case, if we replicate our trial $n$ times and
then describe the resulting dataset statistically, we will obtain a
statistical description not only of the data but of the temperature
measured.  That's because each datum in the statistical dataset has a
determinate meaning, which can be expressed in terms of
representation.

\begin{remark}
  This is of course an oversimplification; the notion of reference or
  denotation with respect to measurement is not clear-cut.  The idea
  here is that a particular measurment yields a number (on a scale)
  that \textbf{refers} to temperature; but does it \textbf{denote}
  \enquote{the} temperature?  Since perfect measurement is not
  possible (all measurements \enquote{contain} error), it cannot be
  the case that the measurement denotes the true temperature.  So
  although the measurement (the number) in some sense refers to the
  temperature of the sample, it does not denote the true temperature.
  This sort of reference is fuzzy.
\end{remark}

Things are quite different for social scientific data, however.

A specific question performance, by contrast, yields an answer, which
is interpretable in terms of discursive norms rather than
correspondence (accuracy of representation).

\subsection{Person}

\subsection{Reference}

The person.  Orthodox view is that answers measure some property or
attribute of the respondent.  One problem with this is that it assumes
a individualist autonomy.  But discursive practice is essentially
social, and the meanings of our sayings cannot be private.

Subjectivity is social, relational; even first-person claims implicate
essentially social stuff, spill over the boundaries of the individual
speaker.

\blockquote{Strongly constitutive sociocultural perspectives in
  psychology have become more developed and influential in recent
  decades, particularly during the past 15 years. These approaches
  include constructionist, discursive, relational, dialogical, and
  neo-Vygotskian theories. They diverge from one another in some
  respects, but are alike in that they all consider psychological
  processes, such as mind and self, to emerge out of social, cultural,
  and historical contexts.\footcite{kirschner_sociocultural_2010}}

\noindent
\cite{bucholtz_identity_2005}\\
\cite{kirschner_sociocultural_2010}\\
\cite{andreouli_identity_2010}

\section{Reliability}

Intuitively, reliability is about replication.  Something is reliable
if we can count on it to \enquote{work} the next time (or every time)
we use it.  It's a kind of prediction.

\begin{remark}
Brandom: reliabilism as a key strategy of representationalism.
Importance of the gerrymandering objection.
\end{remark}

\subsubsection{Statistical Reliability}

\section{Assessment}

The goal is usually stated as \enquote{comparable data}, but what we
really want is \textit{commensurable} data; that is, data that can be
meaningfully statistically analyzed.

\subsection{Validation}

Give a concept of validity, what must we do in order to validate our research?

Standard \sr{} uses several \enquote{methods} of validation.  The best
method involves checking responses against external data, usually
administrative data.  Note that even here the assumption is that the
external data is itself valid and reliable, which is not necessarily
the case.  Administrative data, after all, is just as vulnerable to
error as any other data, and so is the process of obtaining and using
such data for verification purposes.  So even the best method of \sr{}
validation is already fairly weak.

Another common method involves some form of triangulation.  If the
verification method corresponds to correspondence theories of truth,
triangulation methods correspond to coherence theories.  The idea is
simply that the data obtained in an interview should be internally
consistent.  Contradictory responses are taken as indicators of
invalidity.  And so forth.

Regardless of what methods are used, we have no means of truly
establishing validity and reliability.  The problem is that these
concepts implicitly rely on theoretical presuppositions that do not
hold for the human sciences, though they work just fine for the
\enquote{hard} sciences.

A \enquote{valid} question really just means we have \enquote{good} reasons to
believe that we have provided (necessary) and sufficient
\textit{resources} for the respondent to use in producing a correct
and cooperative response.  That is, with the resources we have
provided we have good reason to believe that \R{}s \textit{should}:

\begin{itemize}
\item grasp the intended significance of the question; and
\item cooperatively disclose a correct response
\end{itemize}

This is really the best we can hope for.  Real \enquote{verification} of
answers is not truly possible even for simple non-threatening
questions; for the sort of question we are working with, there is no
means of even approximating verification, since no external source of
information is available.

\subsection{Establishing Reliability}

%%%%%%%%%%%%%%%%%%%%%%%%%%%%%%%%%%%%%%%%%%%%%%%%%%%%%%%%%%%%%%%%
\section{Quality Assurance: Failure, Defect, Cause}

Instead of \enquote{error} and its sources we talk of failure and its
causes.

\subsection{Protocol}
\subsection{Instrument}
\subsection{Interviewer}
\subsection{Respondent}
\subsection{Environment}
\subsection{Other Causes of Breakdown}

\section{Validation}

Here the critical question is how we can determine whether or not
\R{}s answer the status question truthfully; that is, whether they
pick the correct box.  The problem is that there is no way even in
principle to verify answers based on externally obtained data.

\subsection{Triangulation}

A standard method of validating answers is by collectiing collateral
information that can be used to check the internal consistency of
answers.  For example, we might ask whether the \R{} is currently
enrolled in school, and compare the response to the claimed
immigration status group.

There are two ways to obtain triangulation data: we can ask such
questions either before or after we ask the immigration status
question.

The problem is that triangulation amounts to a form of disclosure
analysis.  If we can use it to infer status, so can a hostile analyst.
On the other hand, we might use it to determine whether the \R{} is
responding truthfully, rather than to try to detect status.

We do not want to use the response to the GAM question as a means of
inferring status in conjunction with other items in the questionnaire.
On the contrary, for ethical reasons the instrument must be designed
so that such analysis is not possible.

For extra protection we neither asked for nor collected any identifying information.

\section{Quality}

\section{Design}

\clearpage
\begin{appendices}
\section{Validity in Logic}
\section{Bibliography}
%% \addcontentsline{toc}{chapter}{Bibliography}
%% \bibliographystyle{plainnat}
\printbibliography[heading=none]
\end{appendices}


\end{document}
